\section{Quick Start} \subsection{Prerequisite} NekMOR requires Nek5000
\cite{fischer2008nek5000}.  For installation, one could consult
\url{http://nek5000.github.io/NekDoc/quickstart.html}

\subsection{Download and Setup} This chapter is to provide a quick overview for
user on how to setup and run the basic case using NekMOR. The source is
maintained in a github repository and can be downloaded with the git command:
\begin{lstlisting}[language=bash] $ git clone
https://github.com/SEAL-UIUC/NekMOR.git \end{lstlisting} 
After downloading, add the path of the NekMOR directory and list targeted
objective files to .bashrc (or .bash\_profile):
\begin{lstlisting}[language=bash] 
$ export MOR_DIR='the_path_of_the_NekMOR_directory' 
$ export USR='aux.o ana.o conv.o
const.o dump.o ei.o lapack.o legacy.o mpar.o pod.o filter.o qoi.o rom.o read.o
rk.o time.o unit.o tensor.o riesz.o' 
$ source ~/.bashrc 
\end{lstlisting}

\subsection{A Working Example} 
As a first example, we consider the problem of 2D
flow past a cylinder. This is a canonical test case for ROMs because of its
robust and low-dimensional attractor, which is manifest as a von Karman vortex
street for $\rm Re = UD/\nu >34.37$ \cite{2021APS..DFDP08001D}. The Reynolds
number in our test case $\rm Re = 100$ and domain is $\Omega = [-2.5 : 17]D
\times [-5 : 5]D$, with the unit-diameter cylinder centered at $[0, 0]$.

To get started, we first download NekMOR examples to the home directory:
\begin{lstlisting}[language=bash]
$ cd 
$ git clone https://github.com/Ping-Hsuan/NekMOR_Examples.git
\end{lstlisting}
Go to the cyl directory, setup NekMOR and compile:
\begin{lstlisting}[language=bash]
$ cd NekMOR_Examples/cyl
$ genmap # run partioner, on input type cyl
$ $MOR_DIR/bin/linkm
$ makenek cyl
\end{lstlisting}

To get the snapshot for the cylinder case, one could either use Nek5000 to
collect the snapshots, or download the precomputed snapshots. Assuming you are
in the cyl directory, run the following command to download the stored
snapshots: 
\begin{lstlisting}[language=bash] 
$ $MOR_DIR/bin/gsnaps cyl_rect_full 
\end{lstlisting}
This will create a directory data under NekMOR which has the precomputed
snapshot. 

Create a list with path of snapshots:
\begin{lstlisting}[language=bash] i
$ ls $MOR_DIR/data/cyl_rect/cyl0.f0* > file.list
\end{lstlisting}
Create a list with path of fom mean fields for comparing against the
approximated solution:
\begin{lstlisting}[language=bash] i
$ ls $MOR_DIR/data/cyl_rect/avgcyl0.f0*{05,06,07,08,09,10} > fom_avg.list
\end{lstlisting} 
To use the FOM solution at $t=500$ as the initial condition, add
the path of \textit{cyl0.f00500} below PRESOLVE section in cyl.rea:
\begin{lstlisting}[language=bash] i
$ sed "/PRESOLVE/{n;s|.*|${MOR_DIR}\/data\/cyl_rect\/cyl0.f00500|}" cyl.rea
\end{lstlisting}
Run NekMOR:
\begin{lstlisting}[language=bash] 
$ nekbmpi cyl 1 # run NekMOR on 1 ranks in the background 
$ tail logfile  # view solver output 
\end{lstlisting}

\subsection{Viewing the first 2D example}
We use VisIt for data visualization. Assuming that the Nek5000/bin is in the
execution path,  
\begin{itemize}
   \item Run \textit{visnek romcyl} to view the ROM solution.
   \item Run \textit{visnek bascyl} to view the POD basis.
   \item Run \textit{visnek uiccyl} to view the initial condition, where
      uiccyl0.f00001 being the FOM solution, uiccyl0.f00002 being the projected
      solution, and uiccyl0.f00003 being the error between the two.
   \item Run \textit{visnek lapcyl} to view the laplacian of velocity POD
      basis. This field is only computed in flow pass a cylinder case. It is
      used to compute the drag coefficients.
\end{itemize}

\subsection{Getting the drag coefficients}
To get the drag associated with ROM solution on the cylinder, one simply run
\begin{lstlisting}[language=bash] 
$ grep 'drag' logfile
\end{lstlisting}
which will produce lines such as
\begin{lstlisting}[language=bash] 
5.00000000E+02  2.08581895E-01  6.45283485E-01  8.53865380E-01  dragx
5.00000000E+02  1.91399328E-02  4.07880767E-02  5.99280094E-02  dragy
\end{lstlisting}
indicating, respectively, the physical time, the viscous drag, the pressure
drag and the total drag.

\subsection{Getting the relative error in mean field}
To get the relative error in the mean field, one simply run
\begin{lstlisting}[language=bash] 
$ grep 'relative h1  error' logfile
\end{lstlisting}
which will produce lines such as
\begin{lstlisting}[language=bash] 
 relative h1  error:   3.4366084959588877E-006
 relative h1  error:   1.9777249379045290E-006
\end{lstlisting}
indicating, respectively, the error in the approximated mean flield and the
error in the project mena field.

\section{Parameters file} NekMOR in general supports two types of parameters
files, .rea and .mpar. But .rea will not have all the new features due to the
limitations in number of parameters and potential conflicts with Nek5000.
\subsection{.rea}
\subsection{.mpar} The keys are grouped in different sections
and a specific value is assigned to each key. The .par file follows the
structure exemplified below.  
\begin{lstlisting}[mathescape] 
[SECTION]
key = value
$\cdots$
[SECTION]
key = value
$\cdots$ 
\end{lstlisting}

The sections are: 
\begin{enumerate} \item GENERAL \item POD \item QOI \item
COPT \item FAST \item Buoyancy \item FORCING \item FILTER \item EI \item TENDEC
\end{enumerate}

\begin{table}[!h] \caption{GENERAL key in the .mpar file} \centering
   \begin{tabular}{|c|c|c|} \hline \multicolumn{1}{|c|}{Key} & Value(s) &
      Descripton \\ \hline mode                      &
      \makecell{ALL\\OFF\\ON\\ONB\\AEQ}      & \makecell[l]{MOR mode: \\
   \begin{minipage}{4in} \vskip 4pt \begin{enumerate} \item ALL: \textbf{Build}
         the ROM operator and \textbf{run} the ROM.  \item OFF: \textbf{Build}
            the ROM operator only.  \item ON: \textbf{Run} the ROM with
   operator only.  \item ONB: \textbf{Run} the ROM with reduced basis at hand.
   \item AEQ: \end{enumerate} \vskip 4pt \end{minipage}}  \\ \hline field
      & \makecell{VT\\V\\T}    &  Fields to be solved  \\ \hline nb
      &  $<$int$>$       & Number of modes for the ROM      \\ \hline ns
      &  $<$int$>$       & Number of snapshots used for POD \\ \hline nskip
      &  $<$int$>$       & Skipping interval in snapshots   \\ \hline avginit
      &  $<$int$>$       & Starting time-step of ROM averaging \\ \hline rktol
      &  $<$real$>$      & Adaptive Runge-Kutta error tolerance \\ \hline
      tneubc                    &  \makecell{(no)\\yes}        &  Inhomogeneous
      Neumann BC for temperature          \\ \hline nplay                     &
      $<$int$>$       & Number of replay modes           \\ \hline decoupled
      &  \makecell{(no)\\yes}        &            \\ \hline cflow
   &  \makecell{(no)\\yes}        &            \\ \hline \end{tabular}
\end{table}

\begin{table}[!h] \caption{POD key in the .mpar file} \centering
   \begin{tabular}{|c|c|c|} \hline \multicolumn{1}{|c|}{Key} & Value(s) &
Descripton \\ \hline type & \makecell{(L2)\\H10\\HLM}      & POD-inner
product \\ \hline
mode0                     & \makecell{AVG\\(STATE)}    &  Zeroth mode type\\ \hline
combined                  & \makecell{(no)\\yes}    &  Combine velocity and temperature POD modeds\\ \hline
ratio                     &  $<$real$>$       & Gramian ratio \\ \hline
\end{tabular}
\end{table}

\begin{table}[!h] \caption{QOI key in the .mpar file} \centering
   \begin{tabular}{|c|c|c|} \hline \multicolumn{1}{|c|}{Key} & Value(s) &
Descripton \\ \hline
freq & $<$real$>$      & Frequency of qoi dump ($<1 \rightarrow$ iostep) \\ \hline
tke  & \makecell{(no)\\yes}    & Turbulence kinetic energy \\ \hline
drag & \makecell{(no)\\yes}      & Drag based on OBJ data \\ \hline
nu   & $<$int$>$       & Nusselt number based on problem type (1-3) \\ \hline
\end{tabular}
\end{table}
