\section{Reynolds-averaged Navier-Stokes equations (RANS)}
%
\begin{align}
   \rho \left(\pp{\bU}{t} + \bU \cdot \nabla \bU\right) & = - \nabla \op + \nabla
\cdot \left(\mu (\nabla \bU + (\nabla \bU)^T) + \btau \right), \label{eq:amom}\\
   \nabla \cdot \bU & = 0, \label{eq:acont} \\ 
   \pp{T}{t} + \bU \cdot \nabla T & =  \nabla
\cdot \left(\alpha \nabla T + \mathbf{H}^R \right), \label{eq:aene}
\end{align}
%
where $\btau=-\rho\obuup$ is the Reynolds stress and $\mathbf{H}^R =
-\overline{\bu'T'}$ is the turbulent heat flux.
Eqs.~(\ref{eq:amom}--\ref{eq:aene}) have five unkowns and we close the RANS
equations by modeling the Reynolds stress $\btau$ and turbulent heat flux $\mathbf{H}^R$.

In this work, we follow the Boussinesq hypothesis to model the Reynolds stress
with the eddy viscosity $\nu_t$,
%
\begin{align}
   - (\obuup) = \nu_t \left(\nabla \bU + (\nabla \bU)^T \right) -
\frac{2}{3} k \mathbf{I}, \label{eq:evm}
\end{align}
%
where $2/3 k \mathbf{I}$ is to ensure the correct sum of normal stresses.
\footnote{\begin{equation} \rho
(\overline{(u')^2}+\overline{(v')^2}+\overline{(w')^2}) = -2\rho k.
\end{equation}} 
The eddy viscosity $\nu_t$ could either be a function of one
($\tilde{\nu}$ in Spalart-Allmaras) or two variables ($k$ and $\epsilon$ or
$\omega$).

The simple gradient diffusion hypothesis (SDGH) 
%
\begin{equation}
   \overline{\bu'T'} = -\frac{\nu_t}{\rm Pr_t}\nabla T, \label{eq:sdgh}
\end{equation}
%
and the generalized gradient diffusion hypothesis (GGDH)\footnote{Make sure the
operator form (\ref{eq:ggdh}) matches with the index form $ -C_s
\overline{\bu'_j\bu'_k} \frac{k}{\epsilon} \pp{T}{x_k}$}
%
\begin{equation} 
   \overline{\bu'T'} = -C_s \obuup \frac{k}{\epsilon} \nabla T,
\label{eq:ggdh}
\end{equation}
%
are the two hypothesis \cite{hsieh2004numerical} commonly used to model the
turbulent heat flux.

\section{Galerkin Formulation for the Full-Order Model (FOM)} \label{galerkin_fom}
\noindent The FOM is based on the spectral element method (SEM) in the
open-source code, Nek5000, and uses the $P_{q}$--$P_{q-2}$ velocity-pressure
coupling \cite{maday1987well}, where the velocity is represented as a
tensor-product Lagrange polynomial of degree $q$ in the reference element $\Oh
:= [-1,1]^2$ while the pressure is of degree $q-2$.  The solution in $\Omega =
\bigcup_e \Omega^e$ consists of local representations of $\bu$, $p$, and $T$
that are mapped from $\Oh$ to $\Omega^e$, $e=1,\dots,E$. 

For any $\bu(\bx,t)$, we have a corresponding vector of basis coefficients
$\buu = [\bu_1\dots \bu_{\cN}]^T$ such that 
%
\begin{equation} \label{eq:ui} 
   \bu(\bx,t) = \sum_{j=1}^\cN \bu_j(t) \phi_j(\bx) \in \bX_0^\cN \subset
   \cbH^1_0,
\end{equation}
%
with $\phi_j(\bx)$ the underlying spectral element basis functions spanning the
FOM approximation space, $\bX_0^\cN$.  Because the SEM is nodal-based, each
$\bu_j(t)$ represents the two velocity components at grid point $\bx_j$ in the
spectral element mesh at time $t$.  Similarly, the temperature is given by
%
\begin{equation} \label{eq:ti}
    T(\bx,t) = \sum_{j=1}^\cNb T_j(t) \phi_j(\bx) \in X^{\cN}_0 \subset \cH^1_0.
\end{equation}  Here, $\cH^1$ is the set of square-integrable
%
functions on $\Omega$ whose gradient is also square-integrable and $X^{\cN}
\subset \cH^1$ is the finite dimensional SEM approximation space spanned by $\{
   \phi_j(\bx) \}$. $\cH^1_0$ is the set of functions in $\cH^1$ that vanish
wherever Dirichlet conditions associated with (\ref{eq:ade}) are applied on the
domain boundary $\dO$ and $\cH^1_b$ is the set of functions in $\cH^1$ that
satisfy the prescribed Dirichlet conditions for temperature.  Bold-face
indicates that the space is spanned by vector-valued functions having $d$
components ($d=2$ or $3$) and, in the case of $\bX_0^{\cN} \subset \cbH^1_0$,
that the functions vanish where Dirichlet conditions are applied for
(\ref{eq:PDE}).  The pressure $p$ is in $Y^{\cN} \subset \cL^2(\Omega)$, which
is the space of piecewise continuous functions on $\Omega$ such that
$\int_{\Omega} p^2 \, d\bx < \infty$.  For convenience, we denote $\bbZ^{\cN} =
(\bX^{\cN},Y^{\cN},X^{\cN})$ as the collection of the relevant
finite-dimensional spaces and will add a subscript $0$ or $b$ where required to
explicitly indicate the imposed boundary conditions.

Both the FOM and ROM are cast within the same Galerkin framework.  To begin, we
introduce several inner products for elements in the FOM space, $\bbZ^{\cN}$.
For any pair of scalar fields $(p,q) \in \cL^2(\Omega)$ and $d$-dimensional
vector fields, $\bv(\bx) = [v_1(\bx)\, \dots \, v_d(\bx)]$, $\bu(\bx) =
[u_1(\bx)\, \dots \, u_d(\bx)] $ whose components are also in $\cL^2$, let
%
\begin{equation} \label{eq:vari}
   (q,p) \coloneqq \int_{\Omega} \, q \, p \, d\bx, \qquad
   (\bv, \bu) \coloneqq \int_{\Omega} \, \left( v_1 u_1 + \cdots + v_d u_d
   \right) \, d\bx.  
\end{equation}
%
Further, for $S$, $T \in X^{\cN}$ and $\bv$, $\bu \in \bX^{\cN}$, let
%
\begin{align} \label{eq:vari2}
   a(S,T) &\coloneqq \left(\nabla S , \nabla T \right), \qquad \hspace{0.28in}
   a(\bv,\bu) \coloneqq \left(\nabla \bv , \nabla \bu \right), \\
   c(S,\bu,T) &\coloneqq \left(S , \bu \cdot \nabla T \right), \qquad
   c(\bv,\bu,\bw) \coloneqq \left(\bv , \bu \cdot \nabla \bw \right).
\end{align}
%
For the FOM, we consider the (semi-discrete) weak form of (\ref{eq:PDE}),
{\em Find $(\obU,~p,~\bT) \in \bbZ^{\cN}_b$ such that, for all
$(\bv,~q,~S) \in \bbZ^{\cN}_0$,}
%
\begin{align} \label{eq:var1}
   \left( \bv,\pp{\obU}{t} \right) + \nu \, a(\bv,\obU) +
\textcolor{red}{(\nabla \bv, \nu_t \nabla \obU)} - \left(\nabla \cdot \bv , p
\right) &= -c\left( \bv , \obU , \obU \right) + \left( \bv , \bg(\theta_g) \bT
\right), 
   \\[1.6ex] \label{eq:var2}
   - \left(q, \nabla \cdot \obU  \right) \hspace{2.54in} &= 0,
   \\[1.6ex] \label{eq:var3}
   \left( S,\pp{\bT}{t} \right) + \alpha \, a(S,\bT) + \textcolor{red}{(\nabla
S, \frac{\nu_t}{\rm Pr_t} \nabla \bT)} \hspace{0.75in} &= -c\left( S , \obU,
\bT \right).
\end{align}
%
Here, we have introduced $\obU = \bU + \bU_0(\bx)$ and $\bT = T + T_0(\bx)$ as
functions that have been augmented by (potentially trivial) lifting functions,
$\bU_0$ and $T_0$, which are functions of space only.   If these functions
satisfy the (time-independent) boundary conditions, then one can account for
inhomogeneous boundary conditions by moving them to the right-hand side.  In
the case of the ROM, the lifting functions can also provide an initial
approximation to the solution.  In the sequel, our principal unknowns will be
$\bU$ and $T$.

For temporal discretization of (\ref{eq:var1})--(\ref{eq:var3}), we use a
semi-implicit scheme with $k$th-order backward differencing (BDF$k$) for the
time derivative, implicit treatment of the positive-definite dissipation terms,
and $k$th-order extrapolation (EXT$k$) of the advection and buoyancy terms.  We
typically use $k=3$, to ensure that the imaginary eigenvalues associated with
skew-symmetric advection operator are within the stability region of the
BDF$k$/EXT$k$ time-stepper, as discussed in \cite{fischer2017recent}.  Denoting
the solution at time $t^n = \dt \cdot n$ as $(\obU^n,p^n,\bT^n)$, the full
discretization of the FOM reads {\em Find $(\bU^n,p^n,T^n) \in \bbZ^{\cN}_0$
such that, for all $(\bv,q,S) \in \bbZ^{\cN}_0$,}
%
\begin{align} \label{eq:var1a}
   \frac{\beta_0}{\dt} ( \bv,\bU^n ) + \nu \, a(\bv,\bU^n) - \left(\nabla \cdot
   \bv , p^n \right) &= (\bv,\bff^n), \\[1.6ex] \label{eq:var2a} 
   - \left(q, \nabla \cdot \bU^n  \right) \hspace{1.85in} &=
   \left(q, \nabla \cdot \bU_0 \right), \\[1.6ex] \label{eq:var3a}
   \frac{\beta_0}{\dt} \left( S,T^n \right) + \alpha \, a(S,T^n)
   \hspace{0.93in} &=  (S,Q^n).
\end{align}
%
Equations (\ref{eq:var1a})--(\ref{eq:var3a}) represent a linear unsteady
Stokes plus unsteady heat equation to be solved at each time-step $t^n$.  The
inhomogeneous terms comprise the BDF, advection, buoyancy and lifting terms,
%
\begin{align} \label{eq:rhs1a}
   (\bv, \bff^n) &\coloneqq -\sum_{s=1}^k \, \left[ \frac{\beta_s}{\dt} (
\bv,\bU^{n-s} ) + \alpha_s \left( c( \bv , \obU^{n-s} , \obU^{n-s} ) - ( \bv,
\bg(\theta_g) \bT^{n-s}) - \textcolor{red}{( \nabla \bv , \nu_t \nabla
\obU^{n-s})} \right) \right] \nonumber \\ & - \nu \, a(\bv, \bU_0), \\
   \label{eq:rhs2a} 
   \left(S, Q^n \right) &\coloneqq -\sum_{s=1}^k \, \left[ \frac{\beta_s}{\dt}
(S, T^{n-s}) + \alpha_s \left( c(S, \obU^{n-s}, \bT^{n-s}) -
\textcolor{red}{(\nabla S, \frac{\nu_t}{\rm Pr_t} \nabla \bT^{n-s})} \right)
\right] - \alpha \, a(S,T_0).
\end{align}
%
Here, the $\beta_s$s and $\alpha_s$s are the respective $s$th-order BDF and
extrapolation coefficients for the BDF$s$/EXT$s$ time-stepper
\cite{fischer2017recent}.  Note that the right-hand side of (\ref{eq:var2a})
will be zero if $\bu_0$ is divergence free or at least satisfies the weak
divergence-free condition (\ref{eq:var2}).

Under the assumption that  $\nabla \cdot \bU_0 = 0$, the compact matrix form
\cite{maday1987well, maday1992pn, ronquist1987legendre} for
(\ref{eq:var1a})--(\ref{eq:rhs2a}) is
%
\begin{align} \label{eq:stokes}
   \left[ \begin{array}{cc} \bH & -\bD^T  \\[1.30ex] -\bD&  0
   \end{array}\right] \, \left( \begin{array}{c} \bUu^n \\[1.30ex] \up^n
   \end{array} \right) &= \left(\begin{array}{c}{\hat \buf(\obU^n,\bT^n;
   \theta_g)} \\[1.30ex]0\end{array}\right), \\[2ex]
   \label{eq:heat}
   H_{\alpha} \, \uT^n &= {\hat \uQ(\obU^n,\bT^n)}.
\end{align}
%
Here, $\bUu^n$, $\up^n$, and $\uT^n$ are the vectors of spectral element basis
coefficients.  The corresponding block matrices are
%
\begin{equation}  \label{eq:mat1}
   \bH = \left[ \begin{array}{cc} H_{\nu} & \\ & H_{\nu}
   \end{array} \right], \hspace{.3in}
   \bD = \left[ D_1 \;\; D_2 \right], \hspace{.3in}
\end{equation}
%
with $H_{\nu} = \frac{\beta_0}{\dt} M + \nu A$ and $H_{\alpha} =
\frac{\beta_0}{\dt} M + \alpha A$, with matrices $M$ and $A$ defined below.
The velocity data vectors are ${\hat \buf(\obU^n,\bT^n; \theta_g)}= [{\hat
\uf_1}\;{\hat \uf_2}]^T$, with
%
\begin{align} \label{eq:data1}
   \hat \uf_m & \coloneqq - \sum_{s=1}^k \left[ \frac{\beta_s}{\dt} M
\Uu_m^{n-s} + \alpha_s \left( C(\bUou^{n-s}) \Uou_m^{n-s} -
\textcolor{red}{C_R(\edou)\Uou_m^{n-s}} - g_m M \ubT^{n-s} \right) \right]
\nonumber \\ & - \nu A \Uu_{m,0}, \quad m=1,2.
\end{align}
%
where $g_1 = \cos(\theta_g)$ and $g_2=\sin(\theta_g)$ represent
the parametric forcing.  The thermal load in (\ref{eq:heat}) is
%
\begin{equation}
   \hat \uQ(\bUou^n,\bT^n) \coloneqq - \sum_{s=1}^k \left[ \frac{\beta_s}{\dt}
M \ubT^{n-s} + \alpha_s \left( C(\bUou^{n-s}) {\ubT}^{n-s} -
\textcolor{red}{C_R(\frac{\edou}{\rm Pr_t})\ubT^{n-s}} \right) \right] - \alpha
A \uT_0.
\end{equation}
%
Entries of the respective stiffness, mass, convection, and
gradient matrices are
%
\begin{align}
   \label{eq:A} A_{ij} &= \int_{\Omega} \nabla \phi_i \nabla \phi_j\, d\bx,
   \\
   \label{eq:M} M_{ij} &= \int_{\Omega} \phi_i \phi_j\, d\bx, \\ 
   \label{eq:C} C_{ij}(\bw) &= \int_{\Omega} \phi_i\cdot( \bw \cdot \nabla)
   \phi_j \, d\bx, \\ 
   \label{eq:D} D_{m,ij} &= \int_{\Omega} \psi_i \pp{\phi_j}{x_m} \,d\bx,
   \quad m=1,2,\\
   \label{eq:C_R} C_{R,ij}(w) &= \int_{\Omega} \nabla \phi_i w \nabla
   \phi_j \, d\bx. 
\end{align}
%
Note that $\{ \phi_i(\bx)\}$ forms the spectral element
velocity/temperature basis while $\{ \psi_i(\bx)\}$ constitutes
the pressure basis.

\section{Galerkin Formulation for the Reduced-Order Model (ROM)}\label{galerkin_rom}
\noindent Within the Galerkin framework of the preceding section it is
relatively straightforward to develop a ROM.  One defines a set of functions
$\bzeta_j(\bx) \in \bX^N \subset \bX^{\cN}$, $\theta_j(\bx) \in X^N \subset
X^{\cN}$ such that the coarse-space (ROM) solution, is expressed as
%
\begin{equation} \label{eq:romu}
   \bUou_\tc(\bx) = \sum_{j=0}^N \bzeta_j(\bx) U_{\tc,j}, \qquad
   \bT_\tc(\bx) = \sum_{j=0}^N \theta_j(\bx) T_{\tc,j}.
\end{equation}
%
For the ROM, we insert the expansions (\ref{eq:romu}) into
(\ref{eq:var1})--(\ref{eq:var3}) and require equality for all $(\bv,S)$  in
$\bZ_0^N$. In order to set the boundary conditions, we have augmented the trial
(approximation) spaces $\bX^N$ and $X^N$ with the lifting function $\bzeta_0
\coloneqq \bU_0$ and $\theta_0 \coloneqq T_0$.  The corresponding test spaces,
$\bX_0^N \coloneqq \{\bzeta_j\}^N_{j=1}$ and $X_0^N \coloneqq
\{\theta_j\}^N_{j=1}$, satisfying homogeneous boundary conditions, as is
standard for Galerkin formulation.  The coarse space $\bZ_0^N :=
(\bX_0^N,X_0^N)$ is typically based on a linear combination of full spectral
element solutions of (\ref{eq:stokes})--(\ref{eq:heat}), such as snapshots at
certain time-points, $t^n$, or solutions at various parametric values.  Under
these conditions and with a carefully chosen $\bU_0$, $\bX^N$ is a set of
velocity-space functions that are (weakly) divergence-free and the pressure
terms drop out of (\ref{eq:var1})--(\ref{eq:var2}).  We also note that
$\bzeta_j$ and $\theta_j$ are {\em modal}, not nodal, basis functions.  In this
work, we consider proper-orthogonal decomposition (POD) to construct the
reduced-basis.  The $N$-dimensional POD-space is the space that minimizes the
averaged distance between the snapshot set and the $N$-dimensional subspace of
the snapshots set in the $H^1$ semi-norm.  Further details of the POD basis
selection are provided in \cite{kaneko2020towards} and references therein.

The matrix form for the ROM is readily derived by constructing a
pair of rectangular basis matrices, $\bB$ and $B$, having entries
%
\begin{equation}
   \bB_{ij} = \bzeta_j(\bx_i), \qquad B_{ij} = \theta_j(\bx_i),\label{eq:bb}
\end{equation}
%
where the $\bx_i$s are the spectral element nodal points.  The coarse-system
matrices are $H_{\tc,\nu} = \bB^T \bH \bB$ and $H_{\tc,\alpha} = B^T H_{\alpha}
B$ and the governing system for the ROM becomes
%
\begin{equation} \label{eq:rom}
   H_{\tc,\nu}  \,{\huU_\tc}^n  = \bB^T \, {\hat \buf}(\bUou_\tc^n,\bT_\tc^n;
   \theta_g), \qquad
   H_{\tc,\alpha}\,{\huT_\tc}^n  =  B^T \, {\hat \uQ}(\bUou_\tc^n,\bT_\tc^n).
\end{equation} 
%
We refer (\ref{eq:rom}) as Galerkin ROM (G-ROM) throughout the paper.  The ROM
coefficient vectors, ${\huU}^n_\tc = [U^n_{\tc,1} \dots U^n_{\tc,N}]^T$,
${\huT}^n_\tc = [T^n_{\tc,1} \dots T^n_{\tc,N}]^T$, are determined by solving
the two $N \times N$ linear systems.  Note that the coefficients for the
lifting functions are prescribed: $U_{\tc,0}=T_{\tc,0}=1$.  The initial
coefficients for the ROM are obtained by projecting the initial condition
onto the coarse space $\bZ_0^N$ with the $H^1$ semi-norm, 
%
\begin{equation} \label{eq:romu_intial}
    \huU^0_\tc = \bB^T \bA \bUu^0, \qquad {\hat \uT}^0_\tc = B^T A \uT^0,
\end{equation} 
%
which follows from the fact that the columns of $\bB$ and $B$ are,
respectively, $\bA$- and $A$-orthonormal, where $\bA=$block-diagonal($A$).
To recover the spectral element representation, we simply
prolong the $N$-length vectors $\huU^n_\tc$ and ${\hat \uT}^n_\tc$ with the
set of basis functions and add it with the lifting functions $\bu_0$ and $T_0$
%
\begin{equation} \label{eq:bv}
   \bUou^n_\tc = \bB \huU^n_\tc + \bUu_0, \qquad \ubT^n_\tc =  B {\hat
   \uT}^n_\tc + \uT_0.
\end{equation}
%
The functional representations, $\bUou_\tc^n(\bx)$ and $\bT_\tc^n(\bx)$, are
then obtained from (\ref{eq:ui})--(\ref{eq:ti}).  

The equations for eddy viscosity $\nu_t$ is usually complicated, depending on
several parameters and functions. We here avoid implementing the additional
transport equations and just model the eddy viscosity.

The idea is to expand the eddy viscosity $\nu_t$ as linear combination of POD
basis $\{\xi_i\}$ associated with its snapshots.
%
\begin{equation}
   \nu_t(\bx, t) \approx \sum^N_{i=0}l_i(t)\xi_i(\bx).
\end{equation}
%
To model the eddy viscosity, the remaining question is how one solve for
$\{l_i\}_i$ and there are several strategies:
%
\begin{enumerate} 
   \item In \cite{lorenzi2016pod}, the author considered the coefficient
$l_i(t)$ is same as the velocity.  
   \item In \cite{georgaka2020hybrid, hijazi2020data}, the author considered the
radial basis function \cite{lazzaro2002radial} to interpolate the time
coefficient $l_i(t)$.
\end{enumerate}
%

\subsection{Eddy Viscosity ROM}
We first consider strategy~$1$ to compute the eddy viscosity, 
%
\begin{equation}
   \nu_{\tt, \tc} = \sum^N_{i=0}U_i(t)\xi_i(\bx). \label{eq:eddy_exp}
\end{equation}
%
With Eqs.(\ref{eq:romu}) and (\ref{eq:eddy_exp}), the trilinear term $(\nabla
\bv, \nu_{\tt,\tc} \nabla \obU_c)$ can be expressed as:
%
\begin{equation} 
   \left(\nabla \bv , \nu_{\tt,\tc} \nabla \obU_{\tc}\right) = \left(\nabla
\bv, \sum^N_{k=0}U_k \xi_k(\bx) \nabla (\sum_{j=0}^N \bzeta_j(\bx) U_{\tc,j}) \right)
= \sum^N_{k=0} \sum_{j=0}^N U_{\tc,k} U_{\tc,j} \left(\nabla \bv, \xi_k(\bx) \nabla
\bzeta_j(\bx) \right)
\end{equation}
%
